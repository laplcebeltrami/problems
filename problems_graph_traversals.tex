% This is a simple LaTex sample document that gives a submission format
%   for IEEE PAMI-TC conference submissions.  Use at your own risk.
\documentclass[10pt]{article} %,twocolumn
\usepackage{times,amsmath,amsfonts}

% Use following instead for LaTex 2.09 (may need some other mods as well).
%\documentstyle[times,twocolumn]{article}
\usepackage[dvips]{graphicx,graphics}
% Set dimensions of columns, gap between columns, and paragraph indent
%\setlength{\textheight}{9.5in} \setlength{\textwidth}{7.5in}
%\setlength{\columnsep}{0.3125in} \setlength{\topmargin}{0in}
%\setlength{\headheight}{0.5in} \setlength{\headsep}{-1in}
%\setlength{\parindent}{1pc}
%\setlength{\oddsidemargin}{-.5in}  % Centers text.
%\setlength{\evensidemargin}{-.5in}

% Add the period after section numbers.  Adjust spacing.
\newcommand{\Section}[1]{\vspace{-8pt}\section{\hskip -1em.~~#1}\vspace{-3pt}}
\newcommand{\SubSection}[1]{\vspace{-3pt}\subsection{\hskip -1em.~~#1}
        \vspace{-3pt}}
\newcommand{\bqn}{\begin{eqnarray}}
\newcommand{\eqn}{\end{eqnarray}}
\newcommand{\bq}{\begin*{eqnarray}}
\newcommand{\eq}{\end*{eqnarray}}


\usepackage{url}
\usepackage{hyperref}
\hypersetup{colorlinks,%
citecolor=blue,%
filecolor=blue,%
linkcolor=red,%
urlcolor=blue,%
pdftex}

\newcommand{\blue}[1]{{\color{blue} #1}}
\newcommand{\green}[1]{{\color{green} #1}}
\newcommand{\red}[1]{{\color{red} #1}}

\begin{document}

% Make title bold and 14 pt font (Latex default is non-bold, 16pt)
\title{Problem: Graph Traversals}
% For single author (just remove % characters)
\author{Moo K. Chung\\
%Department of Biostatistics and Medical Informatics\\
%Waisman Laboratory for Brain Imaging and Behavior\\
University of Wisconsin-Madison\\
mkchung@wisc.edu}
% For two authors (default example)
\maketitle \thispagestyle{empty}
 
 The report should be generated in LaTex. Do not use Overleaf and must be able to compile LaTex from your computer. 
 The codes must be written in MATLAB, utilizing a main script \texttt{main.m} that calls the tasks outlined below and produces the corresponding results. Each line of code should include adequate comments for clarity. Sample codes are given (for different project) in \url{https://github.com/laplcebeltrami/PH-STAT}.\\




A graph is denoted as \( G = (V, w) \), where \( V = \{1, \ldots, p\} \) represents a set of vertices consisting of nodes from 1 to \( p \), and \( w = (w_{ij}) \) is the \( p \times p \) edge weight matrix. The edge between nodes $i$ and $j$ is denoted as $(i,j)$. The edge weight \( w_{ij} \) is between nodes \( i \) and \( j \), and $w_{ij} \in [0,1]$. We will assume that the nodes are uniformly distributed along the unit circle. 



\begin{enumerate}


\item Create a MATLAB function \texttt{graph\_circle.m} that takes the edge weight matrix \( w \) as input and displays the graph. Nodes should be depicted as black dots, while the edges should be color-coded based on their respective weights.

\item Implement a function \texttt{graph\_traverse.m} to traverse all nodes with the minimum cost. 
If $E$ is the set of all the edges that the path traversed, the total cost is given by $\sum_{(i,j) \in E} w_{ij}$.
The traversed path should be represented as a matrix. For example,
\[
\begin{array}{ccc}
1 & 5 & w_{15} \\
5 & 7 & w_{57} \\
7 & 8 & w_{78} \\
\end{array}
\]
represents a traversal from nodes \( 1 \) to \( 5 \) to \( 7 \) to \( 8 \).

\item Mathematically prove the correctness of your algorithm. 

\item Write a function \texttt{graph\_traverse\_display.m} that visualizes the traversed path using arrows, overlaid on the output from \texttt{graph\_circle.m}.
\end{enumerate}

Solve the problem with two randomly generated graphs with 10 and 100 nodes respectively.



%\bibliographystyle{biometrics} 
\bibliographystyle{plain}
\bibliography{reference.2023.04.13}

\end{document}

%[mean(bsmean)-1.645*std(bsmean) mean(bsmean)+1.645*std(bsmean)]
